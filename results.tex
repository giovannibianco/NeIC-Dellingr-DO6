
In this \pilot the resources were allocated in a different manner than the first pilot.
In this case the users requested specific resources not in their country through the self-service portal. 
In the first \dell pilot, the users requested an amount of computing time and were manually assigned to an \einfra resource
provider by the \dell project.
Also, the self-service portal expressed the \einfra resources in a common format
``Billing Units" (BU)~\footnote{In general for all \dell resources, 1 CPU
core has been set to cost 0.01 BU per hour and one GPU core cost is 0.1 BU per hour.}.

The \dell \pilot was launched in June~2019 and the last applications were received in December~2019. 
These figures summarize the extent of the Pilot~\footnote{As of \today}:
\begin{itemize}
\item 13 applications were received;
\item \accepted projects were accepted;
\item \BUalloc BU were granted to the accepted projects;
\item \BUcons BU were consumed by the projects.
\end{itemize}
Only two applications were rejected. 
Two of the eleven applications accepted were handled through channels other than the self-service portal.
One was a Danish continuation of resource usage from the first \dell pilot and was handled through the Swedish SUPR~\cite{supr} portal.
The other was am Icelandic continuation of resource usage from the first \dell pilot and was handled through the CSC Service Desk~\cite{csc-service-desk}.

Table~\ref{tab:projects} shows the orders for resources received from users.
\begin{table}
\begin{center}
\rowcolors{2}{gray!25}{white}
\begin{tabular}{|p{7cm}|l|l|r|r|} \hline
\bf Project Title & \bf From & \bf To &\bf Allocated &\bf Used \\
                  & & & {\bf BU} & {\bf BU} \\ \hline
Exploring mechanisms of roughness creation at the nanoscale using a systematic set of large time and length scale Molecular Dynamics simulations & DK & EST & 500 & {2085}\\

Acetonitrile parametrization & DK & FI & 500 & 484 \\

Local structural correlations in Sn doped BCZT ferroelectric relaxors & DK & SE & 3000 & 1569\\

GETM high-resolution modelling for the Baltic Sea & EE & SE & 500 & 594 \\

%% Simulating Electronic and Nuclear Dynamics in Dye-Sensitized Solar Cells & FI & EST & 2000 & \\

Simulation of ions in solutions using QM/MM method & IS & FI & 1000 & 4 \\

Simulating Electronic and Nuclear Dynamics in Dye-Sensitized Solar Cells & IS & FI & 500 & 14 \\

Benchmark comparison of pure, hybrid, double-hybrid DFT functionals against the highly accurate wavefunction methods for reliability for quantum dynamics simulations & IS & FI & 500 & 4 \\

Configuration space sampling of solvated hexaaquairon(II) and hexaaquairon(III) & IS & FI & 500 & 61 \\

Polarizable Embedding of Ions in Solution & IS & FI & 500 & 0 \\

Fluid reactions mechanism on mineral surface in planet interior & IS & SE & 2000 & n/a \\

Atomistic insight into difference and similarities of copper sulfide and oxide corrosion films & SE & FI & 2500 & 1609 \\

\bf Total & & & \BUalloc & \BUcons \\\hline
\end{tabular}
\caption{The accepted projects in the \pilot. (DK=Denmark, EE=Estonia, FI=Finland, IS=Iceland, NO=Norway, SE=Sweden). \label{tab:projects}}
\end{center}
\end{table}
As can be seen in Table~\ref{tab:projects} the amount of BUs allocated varies.
The amount of BUs consumed by each project also varies from very little (zero) to multiples of the original allocation.
One project, with a BU consumption of "n/a" was not correctly assigned an allocation due to a misunderstanding in the project setup stages.
% Overall, the rate of BUs consumed is rather low.

Table~\ref{tab:results} shows the resource allocations between client and hosting countries.
\begin{table}[ht]
\begin{center}
% \vspace{0.5cm}
\rowcolors{2}{gray!25}{white}
\begin{tabular}{|l|r|r|r|r|r|r|r|r|} \hline
{\bf Host/Guest} & \bf Denmark & \bf Estonia & \bf Finland & \bf Iceland & \bf Norway$^*$ & \bf Sweden 
& \multicolumn{2}{c|}{\bf Total} \\ \cline{7-9}
 & & & & & & & \multicolumn{1}{r}{\bf BU} & {\bf \#} \\ \hline
\bf Denmark & x & & & & & & 0 & 0\\
\bf Estonia & 500 & x & & & & & 500 & 1 \\
\bf Finland & 500 & & x & 3000 & & 2500 & 6000 & 7 \\
\bf Iceland & & & & x & & & 0 & 0\\
\bf Norway$^*$  & & & & & x & & 0 & 0\\
\bf Sweden  & 3000 & 500 & & 2000 & & x & 5500 & 3 \\
\hline
\bf Total BU (\#) & 4000 (3) & 500 (1) &  & 5000 (6) &  & 2500 (1) & \BUalloc & \accepted \\ \hline
\end{tabular}
\caption{The number of BUs allocated and hosted per country. The horizontal (vertical) view gives the number of BUs hosted (allocated) per country. $*$ Norway has observer status only.\label{tab:results}}
% \vspace{0.5cm}
\end{center}
\end{table}
As the number of accepted projects was \accepted and the number of entries in Table~\ref{tab:results} is 8, it can be seen that most countries managed to generate~1~request.
The exception to this was Iceland that generated 6 projects and, to a lesser extent, Denmark that generated 3 projects.
It can be seen in Table~\ref{tab:results} that two countries (Finland and Sweden) hosted the majority of
the projects and BUs and one country (Iceland) generated the 
majority of the projects but not the majority of the BUs allocated.

% \todo[inline]{PN: In Table~\ref{tab:projects} and Table~\ref{tab:results}, all numbers are the allocated resources (and not requested) \\
% PN: In Table~\ref{tab:projects}, should Estonia be EST (as in the table) or EE (as in the caption)? \\
% PN: In Table~\ref{tab:projects}, should we round off usage numbers? \\
% PN: Remember to finally update Total allocated (BUalloc) and Total used (BUcons) in main.tex.}
