In this \pilot the resources were allocated in a different manner than the first pilot.
In this case the users requested specific resources not in their country through the portal. 
In the first pilot, the users requested an amount of computing time and were assigned to a resource manually by the \dell project.
Also, the portal expressed the \einfra resources in a common format ``Billing Units" (BU)~\footnote{In general for all \dell resources, 1 CPU core has been set to cost 0.01 BU per month and one GPU core cost is 0.1 BU per month.}.

The \dell \pilot was launched in June~2019 and the last applications were received in December~2019. 
These figures summarize the extent of the Pilot:
\begin{itemize}
\item 12 applications were received;
\item 10 projects were accepted;
\item XX BU were granted to the accepted projects;
\item YY BU were consumed by the projects.
\end{itemize}
Only two applications were rejected. 
Table~\ref{tab:projects} shows the orders for resources received from users.
\begin{table}
\begin{center}
\rowcolors{2}{gray!25}{white}
\begin{tabular}{|p{7cm}|l|l|r|r|} \hline
\bf Project Title & \bf From & \bf To &\bf Requested &\bf Used \\
                  & & & BU & BU \\\hline
Fluid reactions mechanism on mineral surface in planet interior. & IS & SE & 2000 &  \\
Benchmark comparison of pure, hybrid, double-hybrid DFT functionals against the highly accurate wavefunction methods for reliability for quantum dynamics simulations. & IS & FI & 5000 & \\
Simulation of ions in solutions using QM/MM method. & IS & FI & 10000 & \\
Acetonitrile parametrization. & DK & FI & 50000 & \\
Atomistic insight into difference and similarities of copper sulfide and oxide corrosion films. & SE & FI & 50000 & \\
GETM high-resolution modelling for the Baltic Sea. & EST & SE & 500 & \\
Exploring mechanisms of roughness creation at the nanoscale using a systematic set of large time and length scale Molecular Dynamics simulations. & DK & EST & 500 & {\bf 2085}\\
Simulating Electronic and Nuclear Dynamics in Dye-Sensitized Solar Cells. & FI & EST & 2000 & \\
Configuration space sampling of solvated hexaaquairon(II) and hexaaquairon(III). & IS & FI & 2000 & \\
Polarizable Embedding of Ions in Solution. & IS & FI & 2000 & \\
\bf Total & & & 124000 & \\\hline
\end{tabular}
\caption{Another convincing explanation.(DK=Denmark, EE=Estonia, FI=Finland, IS=Iceland, NO=Norway, SE=Sweden). \label{tab:projects}}
\end{center}
\end{table}
As can be seen in Table~\ref{tab:projects} the amount of BUs requested varies over two orders of magnitude.
Table~\ref{tab:results} shows the resource requests between client and hosting countries.
\begin{table}[ht]
\begin{center}
% \vspace{0.5cm}
\rowcolors{2}{gray!25}{white}
\begin{tabular}{|l|r|r|r|r|r|r|r|} \hline
\bf Host/Guest &Denmark &Estonia &Finland &Iceland &Norway$^*$ &Sweden &Total \\\hline
Denmark & x & & & & & & 0 \\
Estonia & 500 & x & & & & & 500 \\
Finland & 50000 & & x & 21000 & & 50000 & 121000 \\
Iceland & & & & x & & & 0 \\
Norway$^*$  & & & & & x & & 0 \\
Sweden  & & 500 & & 2000 & & x & 2500 \\
\hline
Total & 50500 & 500 & 0 & 23000 & 0 & 50000 & 124000 \\ \hline
\end{tabular}
\caption{A decent explanation. \label{tab:results}}
% \vspace{0.5cm}
\end{center}
\end{table}
As the number of accepted projects was 10 and the number of entries in Table~\ref{tab:projects} is 6, it can be seen that most countries only managed to generate 1 request.
The exception to this was Iceland that generated 6 projects and, to a lesser extent, Denmark that generated 2 projects.
