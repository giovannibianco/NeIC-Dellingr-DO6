{
The responses from the participants of the \pilot are given below.
The identities of the participants are intentionally removed.

\subsection*{Responses 1}
\begin{verbatim}
- If you are a user of both the first and second Dellingr pilots... How
     does the experience compare between the two pilots?

I wasn't. 

- How easy was it to get the account on the "foreign" shared resources?

To be perfectly honest, I found the infrastructure around getting accounts and project management 
etc slightly byzantine. But it probably wasn't helped by the fact that I kinda got stuck with two 
accounts for some reason related to my dual employment at two different universities, both using 
the wayf login. What I'm saying is, it was 50%-75% my own fault :) 

- Did you need application support? If so, did it work well?

No.

- Did you have licensing issues? Were they solved?

No issues

- If your project did not use all the allocated resources, why was this
     the case?

I think I used almost all of them, if not all. 

- Other comments on the application framework and/or process?


None i can think of currently. Thanks for letting me be  part of the pilot!
\end{verbatim}
\subsection*{Responses 2}
\begin{verbatim}
- If you are a user of both the first and second Dellingr pilots... How does the experience
compare 
between the two pilots?

I have been using resources from CSC/Finland in both pilots. The experiences have been rather
similar, in both cases positive throughout. Their are two large differences between the rounds 
are: 
i) the new application system for the latest round, which was a bit tricky to understand at
first but the platform seems to have good future potential; 
ii) the new computer cluster at csc (puhti), which works very well.

- How easy was it to get the account on the "foreign" shared resources?
It was easy the first round, and this time I already had an account making it very smooth.

- Did you need application support? If so, did it work well?
Not this round. Last round we had minor issues that the support could solve quickly

- Did you have licensing issues? Were they solved?
Not this round. Last round we needed to confirm our VASP license, which took a few days. 
This is however very common for this program.

- If your project did not use all the allocated resources, why was this the case?
We have not yet had time to use all, but we will.

- Other comments on the application framework and/or process?
Not at this point, other than that is has been working well and that I consider this a very useful 
and good initiative
\end{verbatim}
\subsection*{Responses 3}
\begin{verbatim}

- If you are a user of both the first and second Dellingr pilots... How does the experience
compare between the two pilots?

I did not participate to the first Dellingr pilot.

- How easy was it to get the account on the "foreign" shared resources?

The process for setting up an account to use the CSC resources was easy.

- Did you need application support? If so, did it work well?

I did not.

- Did you have licensing issues? Were they solved?

I did not have any licensing issues.

- If your project did not use all the allocated resources, why was this the case?

I did not use all of the allocated resources because the development and implementation part of 
my project took longer than what initially expected.

- Other comments on the application framework and/or process?

My Dellingr project on CSC terminates on 29.01.2020. Would it be possible to extend the end 
of the project so I can continue using the allocated resources? Or should this be handled 
by CSC?
\end{verbatim}
\subsection*{Responses 4}
\begin{verbatim}

"If you are a user of both the first and second Dellingr pilots... How does the experience
compare between the two pilots?"

-> I cannot answer this question because I am a user of the second Dellingr pilot project. 

"How easy was it to get the account on the "foreign" shared resources?"

-> It was easy as I got the computational budget within 48 hours. The application form was also
straightforward and did not require many details. 

- Did you need application support? If so, did it work well?

-> I filled in the online form myself. The manual for completing the application is clear. 

- Did you have licensing issues? Were they solved?

-> No, I do not. 

- If your project did not use all the allocated resources, why was this the case?

-> This case did not apply to me as I just got the resources a month ago. 

- Other comments on the application framework and/or process?

-> Please continue the project as it is really helpful for early-stage researcher. 
\end{verbatim}
\subsection*{Responses 5}
\begin{verbatim}

- If you are a user of both the first and second Dellingr pilots... How does the experience
compare between the two pilots?

We had positive experience with both pilots. It took us some time in the first pilot to setup
things, but afterward everything was very smooth. For the second pilot, it didn't work
initially because of some misunderstanding (it was not clear on the other side if we got the
approval for the second pilot), but it is working now!

- How easy was it to get the account on the "foreign" shared resources?

Since it was the first time from our university (Aarhus University), it took us some time to
create an account and to make it work. But we did not had any problem afterward.

- Did you need application support? If so, did it work well?

Not that much as we only used LAMMPS software which is straightforward to install. 

- Did you have licensing issues? Were they solved?

No as we only used open-source software

- If your project did not use all the allocated resources, why was this
   the case?

I believe we used even more than allocated resources

- Other comments on the application framework and/or process?

It would be nice to hear about future perspectives of the project and how one can access 
these resources. As I mentioned before, it really helped us a lot.
\end{verbatim}

\subsection*{Responses 6}
\begin{verbatim}
- If you are a user of both the first and second Dellingr pilots. How does the experience 
compare between the two pilots?

The dashboard introduced during pilot phase two was a great boon. Applying for, and monitoring resources, 
was more streamlined.
Pilot phase one was however quite easy to apply for and use.

- Did you need application support? If so, did it work well?

Yes, there was some support needed during pilot phase one,
which was resolved quickly.

- Did you have licensing issues? Were they solved?

No licensing issues. All software used was GNU GPL.

- If your project did not use all the allocated resources, why was this the case?

We still have access to and are using resources from pilot phase two.
Resources from pilot phase one were used fully.

- Other comments on the application framework and/or process?

Keep up the good work. Having an option of applying for computational resources through a shared network 
like this is invaluable for method development.
\end{verbatim}
}
